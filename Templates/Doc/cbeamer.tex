%# -*- coding: utf-8 -*-
\documentclass[12pt,hyperref={bookmarks=false},notheorems]{beamer}
\usepackage{beamerthemesplit}
\usepackage{amsmath,mathrsfs,amscd,graphicx}
%\usetheme{CambridgeUS}
\usetheme{Antibes}
%\usecolortheme{sidebartab}
\usefonttheme{structurebold}
\setbeamercovered{dynamic}
 \setbeamertemplate{theorems}[numbered]
 \setbeamertemplate{blocks}[rounded]
  \let\WriteBookmarks\relax
\usepackage[UTF8]{ctex}
\usepackage{ccmap}
  \DeclareMathOperator{\RE}{Re}
 \DeclareMathOperator{\IM}{Im}
 \DeclareMathOperator{\Ind}{Ind}
 \DeclareMathOperator{\Bd}{Bd}
 \DeclareMathOperator{\Id}{Id}
 \DeclareMathOperator{\DIV}{div}
  \DeclareMathOperator{\grad}{grad}
 \DeclareMathOperator{\trace}{tr}
 \newcommand{\me}{\mathrm{e}}
 \newcommand{\mi}{\mathrm{i}}
 \newcommand{\mj}{\mathrm{j}}
 \newcommand{\mk}{\mathrm{k}}
 \newcommand{\dif}{\mathrm{d}}
 \newcommand{\R}{\mathbb{R}}
 \newcommand{\ROM}[1]{\mathrm{\uppercase\expandafter{\romannumeral#1}}}
\theoremstyle{definition}
 \newtheorem{theorem}{\CJKfamily{hei} 定理}
 \newtheorem{lemma}{\CJKfamily{hei} 引理}
 \newtheorem{definition}{\CJKfamily{hei} 定义}
 \newtheorem{corollary}{\CJKfamily{hei} 推论}
  \newtheorem{example}{\CJKfamily{hei} 例}
 \newtheorem{ramark}{\CJKfamily{hei} 注记}
 \newtheorem{proposition}{\CJKfamily{hei} 命题}
 \renewcommand{\proofname}{\CJKfamily{hei} 证明}
    \title[]{
    \LARGE }
    \author{\bf a}
    \institute[a]{\bf a,太原\ 030000\\\vspace{2mm}
    \texttt{email:{\href{mailto:a@163.com}
        {a@163.com}}}}
    \date[]{}
\begin{document}
\begin{frame}
\titlepage
\end{frame}
  \section*{目录}
\begin{frame}{}
    \tableofcontents
\end{frame}
    \section*{}
      \subsection*{}
\begin{frame}[allowframebreaks]{}	
      \begin{block}{}
    
      \end{block}
\end{frame}
%%%%%%%%=================================================================
\section{}
\begin{frame}[allowframebreaks]{}
        
\end{frame}
\section{Introduction}
    \subsection{Overview of the Beamer Class}
\begin{frame}{Features of the Beamer Class}

      \begin{itemize}
      \item<1-> Normal LaTeX class.
      \item<2-> Easy overlays.
      \item<3-> No external programs needed.
      \end{itemize}
\end{frame}
%%%%------------------------------------------------------------------
\subsection{Test}
\begin{frame}{Test}

  \begin{theorem}
    Let $\mathcal S$ be a logical structure with universe~$U$ and let
    $A \subseteq U$. If

    \begin{enumerate}
    \item
      $\mathcal S$ is well-orderable,
    \item
      every finite relation on~$U$ is elementarily definable
      in~$\mathcal S$, and
    \item
      \alert{$\#_{\!A}^n$} is elementarily \alert{$n$}-enumerable in~$\mathcal S$ via a
      relation that \alert{never `enumerates' both $0$ and~$n$},
    \end{enumerate}

    then \alert{$A$ is elementarily definable} in~$\mathcal S$.
  \end{theorem}
    \begin{overprint}
    \onslide<2>
      \begin{corollary}
        If $\chi_A^n$ is $n$-enumerable by a finite automaton, then
        $A$ is regular.
      \end{corollary}

    \onslide<3>
      \begin{corollary}[with more effort]
        If $\chi_A^n$ is $n$-enumerable by a Turing machine, then $A$
        is recursive.
      \end{corollary}
  \end{overprint}
\end{frame}

 %%%%%%%------------------------------------------------------------
\begin{frame}{How Well Can the Cardinality Function Be Enumerated?}

 \begin{block}{Observation}
    For fixed~$n$, the cardinality function $\#_{\!A}^n$
    \begin{itemize}
    \item
      can be \alert{$1$}-enumerated by Turing machines only for \alert{recursive}~$A$,~but\hskip-0.5cm\hbox{}
    \item
      can be \alert{$(n+1)$}-enumerated for \alert{every} language~$A$.
    \end{itemize}
  \end{block}

  \begin{alertblock}{Question}<2->
    What about $2$-, $3$-, $4$-, \dots, $n$-enumerability?
  \end{alertblock}
\end{frame}
%%%%%%%%%%%%%%%%%%%%%%%%%%%%%%%%%%%%%%%%%%%%%%%%%%%%%%%%%%%%%%%%%%%%%%5
\begin{frame}{Why?}

  \begin{block}{First Explanation}<1>
    The weak cardinality theorems hold both for recursion and automata
    theory \alert{by coincidence}.
  \end{block}

  \begin{block}{Second Explanation}<1-2>
    The weak cardinality theorems hold both for
    recursion and automata theory, \alert{because they are
      instantiations of\\ single, unifying theorems}.
  \end{block}

  \vskip1em
  \visible<2->{
    The second explanation is correct.\\
    The theorems can (almost) be unified using first-order logic.
    }
\end{frame}
\section{参考文献}
\begin{frame}[allowframebreaks]{参考文献}
  \begin{thebibliography}
  {888}
  \bibitem{A} A. D. Alexandrov, \textit{Uniqueness theorems for
surfaces in the large I}, Vestnik Leningrad Univ., \textbf{11}(1956), 5-17.

 \end{thebibliography}
\end{frame}
\begin{frame}
  \begin{center}
  \Huge\CJKfamily{hei} 谢谢大家!
  \end{center}
\end{frame}
\end{document}
